\documentclass[]{article}
\usepackage{amsmath}
\usepackage{physics}

%opening
\title{}
\author{}

\begin{document}

\maketitle

\begin{abstract}

\end{abstract}
In two spatial dimensions particles are not limited to being bosons or fermions as they are in three dimensions, but can be particles known as anyons [1-3] which obey exotic exchange statistics. These anyons can be braided around one another, independent of specific path chosen, altering the quantum mechanical state of the system. Anyons can be classified into two types, Abelian and non-Abelian, both of which are predicted [2-5] to be realized in certain fractional quantum Hall (FQH) systems. Currently, two dimensional electronic systems at low temperatures and high magnetic fields are the most robust platforms for exhibiting these FQH states. [Needs more motivation, mention experimental detection of fractional statistics and quantum computation?]

FQH states are labeled by their filling fraction $\nu$, a dimensionless ratio of electron density to magnetic flux. FQH states at $\nu = \frac{1}{m}$, with $m$ odd, are an incompressible fluid with anyonic quasiparticle excitations exhibiting fractional charge $q = \frac{e}{m}$ and fractional statistical phase $\theta = \frac{\pi}{m}$[13-16]. There is a unique FQH state at $\nu = \frac{5}{2}$ that is considered to be the most promising system for demonstrating non-Abelian exchange statistics. Quasiparticle excitations in this state are anyons with charge $q = \frac{e}{4}$ and whose exchange statistics have two parts, an Abelian part with $\theta = \frac{\pi}{8}$ and a non-Abelian Majorana mode [2, 3]. The quasiparticles can pair to create a quasiparticle with charge $q = \frac{e}{2}$ and statistical phase $\theta = \frac{\theta}{4}$, however the tunneling amplitude is much smaller than that of the unpaired quasiparticles [11, 12 BEC] and will be ignored for the rest of this paper.
 
	Exchange statistics of anyonic quasiparticles may be detected through quantum interferometry experiments, where the quasiparticles braid around a group of localized quasiparticles at the center of an interferometer [9, 12, 27]. In a quantum Hall Fabry-Perot interferometer, interference trajectories are realized through coherent transport along edge channels and tunneling across a pair of split-gated constrictions, which act as beam splitters. For $\nu = \frac{5}{2}$, the diagonal resistance through the device varies as 
	\begin{equation}
	\delta R_D \propto \Re{R_{e/4}(-1)^{N_\psi}e^{i\phi}},
	\end{equation}
	where $R_{e/4}$ is the tunneling amplitude for the Abelian quasiparticles, $N_\psi$ is the Majorana mode that may be filled or unfilled, and $\phi$ is the interference phase which is a sum of the Aharonov-Bohm phase due to flux enclosed by the edge and the statistical phase $2\theta$, as a braid around the device is equivalent to two exchanges, for each quasiparticle encircled by the edge. For $\nu=\frac{5}{2}$ the measurement should detect phase-slip events of $\phi = \frac{\pi}{4}$ corresponding to the excitation of an Abelian quasiparticle, $\phi = \pi$ corresponding to change in the Majorana parity, or $\phi = \frac{5\pi}{4}$ as a combination of the two. Similarly, for $\nu = \frac{7}{3}$ the measurement should detect phase-slip events of $\phi = \frac{2\pi}{3}$ as $\nu = \frac{7}{3}$ does not support Majoranas.
	
Our Fabry-Perot interferometer was fabricated from a high mobility, symmetrically doped GaAs/AlGaAs quantum well with an etched diameter of 1.2 $\mu$m and a width between the two constrictions of 400 nm. An SEM image of the interferometer is shown in Fig. 1a. The sample was mounted on a dilution refrigerator and cooled to 10 mK. Diagonal resistance, which isolates the edge channel [QT],   R$_D$, is measured as the voltage bias on the plunger gate, V$_P$, is swept at a steady rate while the magnetic field, B, is held fixed. The bias voltage changes the interferometer area resulting in interference oscillations and excitation of quasiparticles. Fabry-Perot interferometers function in two regimes: Aharonov-Bohm (AB) where the bulk and edge are decoupled, and Coulomb-Dominated (CD) where they are coupled [4-6BEC]. We believe our device primarily operates in the CD regime with strong coupling between the bulk and the edge, and operates under the conditions [12] where the phase-slips are resistant to charge noise.

	Fig. 2 illustrates interference data from an interferometer at $\nu = \frac{7}{3}$, which behaves as $\nu = \frac{1}{3}$ with two additional inert filled Landau levels. The color plot in (a) demonstrates a positive slope which is characteristic of a device in the CD regime [Theory of FB] as well as a magnetic field value of 5.7442 where only Aharonov-Bohm oscillations are visible. Rapid phase slips with $\delta\phi = \frac{2\pi}{3}$, are visible in (b), as well as corresponding fits, for various magnetic field values, reflecting the excitation of an anyonic quasiparticle. In (c) a histogram is generated by counting each phase slip as an event, fitting a Gaussian with a mean of mean of 1.310 and a standard deviation of 0.120. As these quasiparticles are holes, we expect to see a $\delta\phi = -\frac{2\pi}{3}$ or experimentally $\delta\phi = \frac{4\pi}{3}$, which verifies that our experiment coincides with the theory. It is important to note that one data sweep can dominate the histogram, as there can be many phase slip events for one fit.
	 
	For the $\nu = \frac{5}{2}$ state, under ideal conditions where quasiparticles are well separated from each other and the edge, interference manifests itself in the even-odd effect [19, 20]. Interference disappears when there are an odd number of quasiparticles in the interferometer, and reappears for an even number. However, our interferometer area does not allow for such ideal conditions. For a small cavity, Majoranas couple to each other and the zero energy modes split, with a splitting larger than the temperature. The modes freeze into their lowest energy state, and as the coupling to the edge is strong, any lone Majoranas are absorbed into the edge, persevering interference for both even and odd quasiparticles [12].  
	Fig. 3b and 3c illustrate plunger gate sweeps of the data at various magnetic fields with associated fits, $\delta\phi = \pi$ and $\delta\phi = \frac{\pi}{4}$ respectively. The phase slip for the $\delta\phi = \pi$ case seems to be permanent, as the change in parity of the Majorana mode does not fluctuate much on the experimental time scale. Conversely, for $\delta\phi = \frac{\pi}{4}$ the phase slips happen often and mostly in pairs where a quasiparticle is excited and then absorbed. This behavior seems to be in agreement with the idea that the dopant layer is coupling to the charge of the interferometer, changing the quasiparticle count leaving the Majorana mode mostly unaffected [12]. Fig. 3a shows a detailed interference profile with two separate slopes coinciding with the change from the AB regime to the CD regime as the Landau level is filled. While the majority of the device is operating within the CD regime, the histogram in Fig 3d shows two peaks at $\delta\phi = \frac{\pi}{4}$ and $\delta\phi = \pi$, in agreement with the theory. It is important to note that roughly 28\% of the plunger sweeps show $\delta\phi = \frac{\pi}{4}$ phase slips, the count is comparable to that of the Majorana modes, reflecting excitability of single quasiparticles. Similarly with the $\nu = \frac{7}{3}$ case, a single fit within a file can cover many phase slips creating large peaks in the histogram. 
	 

\end{document}
